\chapter*{Abstract}

With the increased use of \gls{5g} personal wireless devices in close proximity to the human body, international exposure limits have been recently updated to take into account exposure above \SI{6}{\GHz}, including the entirety of the \gls{mmw} spectrum.
The limits for local steady-state exposure are given in terms of the \gls{apd} and its free space counterpart -- the \gls{ipd}.
Both quantities require spatial averaging on the control evaluation plane whose surface area depends on the frequency of the incident \gls{em} field.
Spatially-averaged power densities have been proven to be valid proxies for local temperature rise on exposed surfaces with no pronounced curvature.
However, when local curvature radii on the surface are of the same order of magnitude as the wavelength of the incident wave, accuracy of the extracted dosimetric quantities is compromised.
This is most often manifested through the underestimation of actual values, which leads to the inability to demonstrate compliance.
This work presents a literature review that pertains to the current approach in the spatial averaging of power densities for local exposure above \SI{6}{\GHz}.
In addition, proposed potential upgrades in terms of spatial averaging on non-planar surfaces are addressed with a special emphasis on canonical geometries such as the sphere and cylinder.