\chapter*{Sažetak}

S povećanom upotrebom \gls{5g} osobnih bežičnih uređaja u neposrednoj blizini ljudskog tijela, međunarodna ograničenja izloženosti nedavno su ažurirana kako bi uzela u obzir izloženost iznad \SI{6}{\GHz}, uključujući i kompletni \gls{mmw} spektar.
Granice za lokalnu izloženost u stacionarnom stanju definirane su kroz \gls{apd} i pandan u slobodnom prostoru -- \gls{ipd}.
Obje veličine zahtijevaju prostorno usrednjavanje na ravnoj kontrolnoj plohi čija je površina zadana s obzirom na frekvenciju upadnog \gls{em} polja.
Pokazano je da su prostorno usrednjene gustoće snage valjane zamjene za lokalni porast temperature na izloženim površinama bez izražene zakrivljenosti.
Međutim, kada su lokalni polumjeri zakrivljenosti na površini istog reda veličine kao i valna duljina upadnog vala, točnost ekstrahiranih dozimetrijskih veličina je kompromitirana.
Ovo se najčešće očituje kroz podcjenjivanje stvarnih vrijednosti, što dovodi do nemogućnosti iskazivanja usklađenosti.
Ovaj rad predstavlja pregled literature koji se odnosi na trenutačni pristup u prostornom usrednjavanju gustoće snage za lokalnu izloženost iznad \SI{6}{\GHz}.
Osim toga, predložene potencijalne nadogradnje u smislu prostornog usrednjavanja na neplanarnim površinama se obrađuju s posebnim naglaskom na kanonske geometrije kao što su sfera i cilindar.